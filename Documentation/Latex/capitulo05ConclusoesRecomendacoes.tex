\chapter{Conclusões}
\label{cap:05}

O presente trabalho explorou métodos de segmentação de imagens com foco na aplicação do modelo Segment Anything (SAM) em imagens de média e baixa resolução. Foram abordadas diversas técnicas de segmentação, incluindo métodos baseados em camadas e blocos, e foi realizado um estudo aprofundado das ferramentas e bibliotecas em Python, como PyTorch, Scikit-image e OpenCV, que sustentaram o desenvolvimento dos experimentos. A partir das análises realizadas, constatou-se que o SAM oferece bons resultados em imagens de média resolução, com valores de \textit{Mean Squared Error} (MSE) e \textit{Normalized Cross-Correlation} (NCC) indicativos de precisão e consistência. 

No entanto, os resultados em baixa resolução apresentaram maior instabilidade, demonstrando a limitação da IA em reconhecer bordas e detalhes em imagens de menor qualidade. A análise dos tempos de execução também apontou que a implementação do SAM em computadores com suporte a GPU (CUDA) contribui para a eficiência, mas mesmo com hardware moderno, a variabilidade entre tempos sugere limitações na escalabilidade do método para aplicações em tempo real ou em larga escala.

Este estudo contribui para a área de visão computacional ao validar a eficácia do SAM em contextos específicos de resolução, além de propor uma metodologia para análise de segmentação. A aplicação prática desses métodos em sistemas de segmentação de baixa resolução requer melhorias para garantir maior estabilidade e precisão, indicando possíveis caminhos para trabalhos futuros que explorem alternativas híbridas entre SAM e outras técnicas, visando aumentar a eficácia da IA em diversos cenários de imagem. Em síntese, o trabalho revelou a viabilidade e as limitações do SAM, contribuindo para a construção de uma base teórica e prática sobre a aplicabilidade de modelos de segmentação em imagens de baixa qualidade.

Com base nos resultados obtidos e nas limitações identificadas ao longo de todo o atual estudo, algumas direções para trabalhos futuros podem ser apontadas.
Umas delas é investigar técnicas de \textit{depth estimation}. Outra linha promissora é a investigação da identificação de profundidade com base nos grupos de segmentação obtidos. Essa abordagem permitiria a extrapolação de informações tridimensionais a partir de dados bidimensionais, contribuindo para aplicações que demandam análise de cenas em profundidade, como em realidade aumentada, mapeamento dentre outros. A combinação dos grupos segmentados com técnicas de reconstrução ou estimativa de profundidade poderia revelar novos padrões e ampliar a aplicabilidade para médias e baixas resoluções o que pode oferecer a artistas uma maior eficiência quando se trata de elevar o design, ainda mais quando se trata de pixel artes.