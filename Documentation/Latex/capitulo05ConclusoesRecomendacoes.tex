\chapter{Conclusões Parciais}
\label{cap:05}

Foi descrita a modelagem da ferramenta própria, que envolve a criação de uma malha por cima da imagem para definir a profundidade e a variação de cor, permitindo a manipulação da luz. Além disso, foi citado as tecnologias adotadas, como o Visual Studio para programação em \textit{C\#} e a Unity como a game \textit{engine} escolhida.

Para avaliar a eficácia da ferramenta, foi planejado realizar uma avaliação qualitativa dos resultados obtidos, incluindo a análise da alteração de imagens, a viabilidade da ferramenta e a eficácia dos cálculos realizados. Para fim é esperado que a ferramenta esteja concluída mesmo que de forma manual, para que ocasionalmente possa ter um progresso posterior em outros trabalhos.


%São descritas claramente as conclusões retiradas das discussões e dos experimentos realizados no decorrer da pesquisa, e finalizada a parte textual do trabalho. Recomendações são declarações concisas de ações, julgadas necessárias a partir das conclusões obtidas, a serem usadas no futuro. Ou seja, lembre-se de apresentar os possíveis trabalhos futuros derivados do seu trabalho.