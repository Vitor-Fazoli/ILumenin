\chapter*{}
\noindent{\textbf{RESUMO}}

\noindent{A iluminação é de grande importância para pinturas e esboços digitais nos dias de hoje, pois fornece a percepção de volume ao ambiente. Porém, a iluminação ainda é difícil de replicar em uma imagem digital, pois representa algo muito diferente de um ambiente tridimensional, ainda mais se usarmos uma imagem em baixa resolução. Nesse sentido, O objetivo do trabalho é simular a iluminação em imagens digitais de média e baixa resolução por meio do desenvolvimento de uma ferramenta. Para se alcançar o objetivo, pretende-se que a ferramenta considere informações de luz em um espaço tridimensional para simular a iluminação na imagem bidimensional. Portanto o que se espera deste trabalho é que imagens de média e baixa resolução apresentem boa iluminação.}

\vspace{\onelineskip}

\noindent{\textbf{Palavras-chave}: Iluminação. Algoritmo. Ambiente. Bidimensional.}

%Tópicos:
%1) Importância da iluminação para as artes digitais;
%2) Dificuldade de iluminar imagens de média e, principalmente, de baixa resolução;
%3) Objetivo geral do trabalho;
%4) Descrição geral da metodologia;
%5) Considerações parciais enfatizando os resultados esperados.