\chapter*{}
\noindent{\textbf{RESUMO}}

\noindent{
Este Trabalho de Conclusão de Curso apresenta uma pesquisa sobre a segmentação de imagens utilizando o modelo Segment Anything Model (SAM). O objetivo central é avaliar a eficácia do SAM na segmentação de imagens de média e baixa resolução, destacando sua relevância para aplicações diversas na área de visão computacional. A pesquisa inicia-se com uma revisão da literatura sobre segmentação de imagens, abordando as principais técnicas e abordagens existentes. Posteriormente, é detalhada a metodologia empregada no estudo, que incluiu a criação de uma base de imagens composta por 50 pixel artes de diferentes resoluções e contextos visuais. Os resultados obtidos evidenciam que o SAM apresenta um desempenho satisfatório na segmentação de imagens de média e baixa resolução, mesmo em cenários com presença de ruído ou variações significativas de iluminação.
}

\vspace{\onelineskip}

\noindent{\textbf{Palavras-chave}: Segmentação de imagens. Modelo Segment Anything. Média e baixa resolução.}

\break

\noindent{\textbf{ABSTRACT}}

\noindent{This thesis presents research on image segmentation using the Segment Anything Model (SAM). The main objective is to evaluate the effectiveness of SAM in segmenting medium and low-resolution images, highlighting its relevance for various applications in the field of computer vision. The research begins with a literature review on image segmentation, discussing the main existing techniques and approaches. The methodology employed in the study is then detailed, which included the creation of an image database composed of 50 pixel arts with different resolutions and visual contexts. The results obtained show that SAM performs satisfactorily in segmenting medium and low-resolution images, even in scenarios with noise or significant lighting variations.}

\vspace{\onelineskip}

\noindent{\textbf{Keywords}: Image segmentation. Segment Anything Model. Medium and low resolution.}

%Tópicos:
%1) Importância da iluminação para as artes digitais;
%2) Dificuldade de iluminar imagens de média e, principalmente, de baixa resolução;
%3) Objetivo geral do trabalho;
%4) Descrição geral da metodologia;
%5) Considerações parciais enfatizando os resultados esperados.