\chapter {Introdução}
\label{cap:01}

A segmentação de imagens tem se mostrado crucial na visão computacional, desempenhando um papel fundamental em diversas aplicações, como diagnóstico médico, monitoramento ambiental, segurança pública e condução autônoma. A capacidade de dividir uma imagem em regiões distintas permite extrair informações específicas e realizar análises mais precisas. No entanto, essa tarefa é frequentemente desafiada por fatores como ruído, variações de iluminação e baixa qualidade das imagens, especialmente em contextos onde estamos entre as baixas e médias resoluções de imagens.

Além de identificar objetos ou regiões de interesse, a segmentação desempenha um papel preparatório importante em outras etapas de processamento de imagens. No caso da melhoria de iluminação, por exemplo, segmentar as áreas relevantes da imagem primeiro é essencial para aplicar correções direcionadas, minimizando interferências em partes irrelevantes e preservando a integridade visual. Essa abordagem é especialmente valiosa em aplicações que exigem alto nível de detalhe, como análise médica ou inspeção industrial, onde pequenas falhas de iluminação podem comprometer os resultados.

A iluminação, por sua vez, é um problema recorrente em processamento de imagens. Variações de intensidade, sombras e reflexos podem dificultar não apenas a segmentação, mas também qualquer análise posterior. Por isso, estratégias que combinam segmentação com técnicas avançadas de equalização, balanceamento de cores ou iluminação adaptativa têm o potencial de melhorar significativamente a qualidade e a utilidade das imagens processadas. Este trabalho busca explorar como a segmentação inicial pode facilitar esses ajustes, destacando sua importância como etapa crítica em fluxos complexos de processamento.

Com o advento das inteligências artificiais modernas, especialmente os modelos de \textit{deep learning}, surge uma nova perspectiva para resolver essas limitações. Neste contexto, o Segment Anything Model (SAM) emerge como uma tecnologia inovadora, propondo uma abordagem generalista e flexível para a segmentação de imagens.

Desenvolvido com um conjunto de dados extenso de 1 bilhão de máscaras provenientes de 11 milhões de imagens, o SAM representa um avanço significativo na capacidade de identificar e isolar objetos em diferentes contextos. O presente trabalho tem como objetivo principal investigar a eficácia do SAM na segmentação de imagens, com ênfase em imagens de média e baixa resolução. Através de uma análise rigorosa que combina métricas quantitativas como \textit{Mean Squared Error} (MSE) e \textit{Normalized Cross-Correlation} (NCC), busca-se compreender as potencialidades e limitações deste modelo de inteligência artificial.

Diferentemente das abordagens tradicionais que requerem treinamento específico para cada conjunto de dados, o SAM propõe uma metodologia de \textit{zero-shot}, capaz de realizar segmentações precisas sem necessidade de retreinamento. Esta característica o torna particularmente interessante para aplicações que demandam adaptabilidade e eficiência.

Ao explorar as capacidades do SAM, esta pesquisa não apenas avalia sua performance técnica, mas também contribui para a compreensão mais ampla das possibilidades emergentes na intersecção entre inteligência artificial e processamento de imagens. Os resultados obtidos podem fornecer insights valiosos para pesquisadores e profissionais que buscam soluções inovadoras em segmentação de imagens.

\section{Objetivos}

\subsection{Objetivo Geral}
O objetivo do trabalho é avaliar a segmentação de imagens de média e baixa resolução por meio do modelo de inteligência artificial Segment Anything.

\subsection{Objetivos Específicos}
\begin{itemize}
	\item Preparar uma base de imagens de média e baixa resolução.
	\item Preparar imagens de referência para avaliação da segmentação.
\end{itemize}